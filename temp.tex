\documentclass[a4paper]{article}
\usepackage[a4paper,top=3cm,bottom=2cm,left=3cm,right=3cm,marginparwidth=1.75cm]{geometry}
\usepackage{amsmath}
\usepackage{graphicx}
\title{\Huge{\textbf{ASSIGNMENT-1\\ \ \\}}}
\date{\LARGE{\textbf {11th AUG, 2017\\ \ \\}}}
\author{\LARGE{\textbf{Middepogu Manoj:160050075}}\\ \ \\
\LARGE{\textbf{Pavan Bhargav Tangirala:160050076}}}
\begin{document}
\maketitle
\newpage
\begin{flushleft}
\Large{\textbf{\underline{PROBLEM 1}):-}}
Given n values $\{x_i\}_{i=0}^n$ having mean $\mu$, and median $\nu$ and standard deviation $\sigma$ and $n$ is even.\\
\ \\
\underline{R.T.P}:-$\langle\mu-\nu\rangle \underline{<}\sigma.$ 
	\LARGE{
	Let $|\mu-\nu|$ be P.}\\
\begin{center}
  \LARGE{
	\centering{${\langle\mu-\nu\rangle = \Bigg \langle \frac{\sum_{i=1}^{n} x_i}{n} - \nu \Bigg \rangle 
	}$\\	
	 $=>{\langle\mu-\nu\rangle = \Bigg \langle \frac{\sum_{i=1}^{n} (x_i - \nu)}{n}\Bigg \rangle 
	}$}}\\
\end{center}
\ \ \ \ (Here, $<>$ denotes modulus here.)\\ \ \\
\Large{As we know that $|\sum_{i=1}^{n} {(x_i-\nu)}| \underline{<} \sum_{i=1}^{n} |x_i-\nu|$.So,nP is less than $\sum_{i=1}^{n} |x_i-\nu|$ and the equation $\sum_{i=1}^{n} |x_i-x|$ has it's minimum at x=$\nu$.\\
So we can say that  $nP \underline{<} \sum_{i=1}^{n} |x_i-\mu|$.
We Know that R.M.S $\underline{>}$ A.M for any 2 set of Positive No.'s.\\
}
\begin{center}
\LARGE{So,$\sqrt\frac{\sum_{i=1}^{n} (x_i-\mu)^2}{n} >= (\frac{\sum_{i=1}^{n} |x_i-\mu|}{n})$\\

$(i)\sqrt\frac{\sum_{i=1}^{n} (x_i-\mu)^2}{n-1} >= \sqrt\frac{\sum_{i=1}^{n} (x_i-\mu)^2}{n}>=(\frac{\sum_{i=1}^{n} |x_i-\mu|}{n})$\\

$(ii)(\frac{\sum_{i=1}^{n} |x_i-\mu|}{n})>=(\frac{\sum_{i=1}^{n} |x_i-\nu|}{n})$
}\\
\end{center}
So,We can prove that $\sigma >= |\mu - \nu|,$ and the equality happens when all values are equal(from (i)).
\end{flushleft}
\newpage
\begin{flushleft}
\Large{\textbf{\underline{PROBLEM 2}):-}}
\Large{
Given 4 sets of n values ${\{x_i\}}_{i=1}^n$ ; ${\{y_i\}}_{i=1}^n$; ${\{z_i\}}_{i=1}^n$; ${\{w_i\}}_{i=1}^n$ where $z_i=ax_i+b;w_i=cy_i+d.$
}\\
Proof:-\\
Let $\mu_x$ be the mean of ${\{x_i\}}_{i=1}^n$ then $a\mu_x+b$ is the mean of ${\{z_i\}}_{i=1}^n$.\\
Similarly $c\mu_y+d$ is the mean of ${\{w_i\}}_{i=1}^n$ where $\mu_y$ is the mean of${\{y_i\}}_{i=1}^n$.\\
\begin{center}
\LARGE{
$r(z,w)=\frac{\sum_{i=1}^n (z_i-\mu_z)(w_i-\mu_w)}{\sqrt{\sum_{i=1}^n (z_i-\mu_z)^2*\sum_{i=1}^n (w_i-\mu_w)^2}}$
\\ \ \\
$r(z,w)=\frac{\sum_{i=1}^n (ax_i+b-(a\mu_x+b)) (cy_i+d-(c\mu_y+d))}{\sqrt{\sum_{i=1}^n (ax_i+b-(a\mu_x+b))^2*\sum_{i=1}^n (cy_i+d-(c\mu_y+d))^2}}$
\\ \ \\
$r(z,w)=\frac{\sum_{i=1}^n (ax_i-a\mu_x)(cy_i-c\mu_y)}{\sqrt{\sum_{i=1}^n (ax_i-a\mu_x)^2*\sum_{i=1}^n (cy_i-c\mu_y)^2}}$
\\ \ \\
$r(z,w)=\frac{(ac)*\sum_{i=1}^n (x_i-\mu_x)(y_i-\mu_y)}{\sqrt{(ac)^2\sum_{i=1}^n (x_i-\mu_x)^2*\sum_{i=1}^n (y_i-\mu_y)^2}}$
\\ \ \\
$r(z,w)=(ac/|ac|)*\frac{\sum_{i=1}^n (x_i-\mu_x)(y_i-\mu_y)}{\sqrt{\sum_{i=1}^n (x_i-\mu_x)^2*\sum_{i=1}^n (y_i-\mu_y)^2}}$
}\\ \ \\
By definition of correlation coefficient r for all x,y:-\\ \ \\
$r(x,y)=\frac{\sum_{i=1}^n (x_i-\mu_x)(y_i-\mu_y)}{\sqrt{\sum_{i=1}^n (x_i-\mu_x)^2*\sum_{i=1}^n (y_i-\mu_y)^2}}$
\end{center}
$(ac/|ac|)$ is 1 if $ab > 0$\ and\ -1\ if\ $ab\ <\ 0$.\\
So,$r(z,w)=+r(x,y)\ if\ ab\ >\ 0,$ i.e., a and b are of same sign. \\
else $r(z,w)=-r(x,y)\ if\ ab\ < 0,$ i.e., a and b are of different sign.\\ \ \\
\end{flushleft}
\newpage
\begin{flushleft}
\Large{\textbf{\underline{PROBLEM 3}):-}}
\Large{
Given set of n values ${\{x_i\}_{i=1}^n}$ and $\mu$ is the mean and $\sigma$ is the standard deviation.\\ \ \\
R.T.P:-$|x_i-\mu|\underline{<}\sigma\sqrt{n-1}$\\
}
\begin{center}
$\sigma = \sqrt{\frac{\sum_{i=1}^n (x_i-\mu)^2}{n-1}}$\\ \ \\
So,$\sigma\sqrt{n-1}=\sqrt{\sum_{i=1}^n (x_i-\mu)^2}$ \\ \ \\
$\sigma\sqrt{n-1}=\sqrt{{\sum_{j!=i}^n (x_j-\mu)^2}+(x_i-\mu)^2}$\\ \ \\
$(\sigma\sqrt{n-1})^2-(x_i-\mu)^2={\sum_{j!=i}^n (x_j-\mu)^2}$\\ \ \\
As,${\sum_{j!=i}^n (x_j-\mu)^2} >= 0$\\ \ \\
$(\sigma\sqrt{n-1})^2-(x_i-\mu)^2 >= 0$\\ \ \\
So, $\sigma\sqrt{n-1} >= |x_i-\mu|$\\ \ \\
\end{center}
\end{flushleft}
\newpage
\begin{flushleft}
\Large{\textbf{\underline{PROBLEM 4}):-}}
\Large{
Given:-$C_1,C_2,C_3$ are events that the Car is behind doors 1,2,3 respectively.\\
$P(C-i)=\frac{1}{3} , i\epsilon\{1,2,3\}.$\\ \ \\
(i)$Z_i$ be the event that contestant chose door i.\\ \ \\
\begin{center}
$P(C_i|Z_i) = \frac{P(C_i.Z_i)}{P(Z_i)}$ \\ \ \\
\end{center}
As, $C_i\ and\ Z_i$ are independent 
\begin{center}
$P(C_i|Z_i) = \frac{P(C_i)P(Z_i)}{P(Z_i)}$\\ \ \\
$P(C_i|Z_i) = \frac{{\frac{1}{3}}*{\frac{1}{3}}}{\frac{1}{3}}$\\ \ \\
So, $P(C_i|Z_i)=\frac{1}{3}$.
\end{center}
Same for $z = 1$ so,$P(C_i|Z_1)=\frac{1}{3}$ for all $i\epsilon\{1,2,3\}$.\\ \ \\ 
(ii)\\ 
\begin{center}
$P(H_3|C1,Z1)= \frac{1}{2}$.\\ \ \\
$\frac{C}{Z_1} \underline{\ \ }\ \underline{\ \ }$
\end{center}
As the contestant chose door 1 and the car is in door 1 the host has 2 choices {door 2 and door 3} as he doesn't choose the door in which there is a car or the door the contestant chose. \\ \ \\
\begin{center}
$\frac{\ }{Z_1}\ \underline{C}\ \underline{\ \ }$\\ \ \\
$P(H_3|C2,Z1)= 1$\\ \ \\
\end{center}
As the contestant chose door 1 and the car is in door 2 host has only one choice i.e., to go with 3.\\ \ \\
\begin{center}
$\frac{\ }{Z_1}\ \underline{\ \ }\ \underline{C}$\\ \ \\
$P(H_3|C3,Z1)= 0$\\ \ \\
\end{center}
As the contestant chose door 1 and the car is in door 3 host has only one chance i.e., to go with 2(not 3).So,probability that he chooses door 3 when car is in door 3 and contestant chose door 1 is Zero.\\ \ \\ 
(iii)\\ \ \\
\begin{center}
$P(C_2|H_3,Z_1)=\frac{P(H3|C_2,Z_1)*P(C_2 Z_1)}{P(H_3 Z_1)}$\\ \ \\
We have calculated the value of ${P(H3|C_2,Z_1)}$ is 1 in (ii) and $P(C_2 | Z_1)$ is $\frac{1}{9}$ as $C_2\ and\ Z_1$ are 2 independent events so  $P(C_2 | Z_1)\ is\ equal\ to\ P(C_2)*P(Z_1)$ i.e., $\frac{1}{9}$.\\ \ \\
$\frac{\ }{Z_1}\ \underline{H}\ \underline{H}\ \ |\ \ \ \underline{H}\ \frac{\ }{Z_1}\ \underline{H}\ \ |\ \ \underline{H}\ \underline{H}\  \frac{\ }{Z_1}$.\\ \ \\
\end{center}
These are the 3 cases possible for contestant and host for selecting doors respectively .So totally there are 6 cases and probability that host chooses door3 and contestant choosing door 1 is $\frac{1}{6}$ so $P(H_3\ Z_1)\ is\ \frac{1}{6}$\\ \ \\
So, the value of $\frac{P(H3|C_2,Z_1)*P(C_2 Z_1)}{P(H_3 Z_1)}$ is $\frac{1*1/9}{1/6}\ =\ \frac{2}{3}$.\\ \ \\
%%%%%%
(iv)\\ \ \\
\begin{center}
$P(C_1|H_3,Z_1)=\frac{P(H3|C_1,Z_1)*P(C_1 Z_1)}{P(H_3 Z_1)}$\\ \ \\
$P(C_1|H_3,Z_1)\ =\ \frac{1}{2}$ (by ii).
$P(C_1\ Z_1)=\frac{1}{9}$\\ \ \\
As,$C_1\ and\ Z_1$ are independent $P(C_1 Z_1)=P(C_1)*P(Z_1)=\frac{1}{9}$ and $P(H_3,Z_1)=\frac{1}{6}$ as we have calculate in (iii).\\ \ \\
So, $\frac{P(H3|C_1,Z_1)*P(C_1 Z_1)}{P(H_3 Z_1)} = \frac{\frac{1}{2}*\frac{1}{9}}{6}\ =\ \frac{1}{3}$.\\ \ \\
\end{center}
(v)\\ \ \\
As,$P(C_2|H_3,Z_1)>P(C_1|H_3,Z_1)$ we can say that Shifting is beneficial to win the Car.
where $P(C_2|H_3,Z_1)$ and $P(C_1|H_3,Z_1)$ meanings remain as mentioned in the question.\\ \ \\
}
\end{flushleft}
\newpage
\begin{flushleft}
\Large{\textbf{\underline{PROBLEM 5}):-}}
\Large{
Given: There is an island where people speak one or more of $n$ different languages.The proportion of people speaking those languages is $p_1,p_2,....,p_n$ where for all i $0\underline{<}p_i\underline{<}1$ .\\ \ \\
The proportion of people who can speak exactly one language, assuming the
ability to speak different languages are all independent is\\ \ \\
\begin{center}
$\sum_{i=0}^n{p_i(1-p_{i+1})....}$\\ \ \\
$=\ {\prod_{i=1}^n ({1-p_i})}(\frac{p_1}{1-p_1}+\frac{p_2}{1-p_2}+....)$\\ \ \\
$=\ \prod_{i=1}^n ({1-p_i})(\sum_{i=1}^n {\frac{p_i}{1-p_i}})$\\ \ \\
Note:- Because all languages are independent we can say portion that are speaking only language $p_i$ is $p_i*(1-p_{i+1})*....$ i.e., in terms of probability people speaking $p_i$
 for every i and not speaking any other language for all j except i.So,we get the above summation.\\ \ \\ 
\end{center}
}
\end{flushleft}
\end{document}